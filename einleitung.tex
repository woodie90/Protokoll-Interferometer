\section{Einleitung}

Der Laserstrahl kann näherungsweise als ebene TEM-Welle beschrieben werden. Deren Feldgrößen erfüllen die eindimensionale homogene Wellengleichung
	\begin{align*}
	\left(\frac{1}{v^2}\frac{\partial^2}{\partial t^2} - \frac{\partial^2}{\partial z^2}\right)\psi(z,t).
	\end{align*}
Die Lösungen
	\begin{align*}
	E_x(z,t) &=E_{0}\cos(\omega t - kz + \varphi)\\
	H_y(z,t) &=\frac{E_{0}}{Z}\cos(\omega t - kz + \varphi)
	\end{align*}
beschreiben eine monochromatische ebene Welle, die sich in $+z$-Richtung ausbreitet. Ihre Parameter sind
	\begin{align*}
	& f=\frac{\omega}{2\pi} &&\lambda=\frac{2\pi}{k} && v = \frac{\omega}{k} = c_0 \mbox{ : Vakuumlichtgeschwindigkeit.}
	\end{align*}
Bei der Überlagerung zweier Wellen gleicher Frequenz und Polarisation kommt es zu stationärer Interferenz. Die Amplitude der resultierende Feldstärke kann durch ungestörte Superposition bestimmt werden:
	\begin{align*}
	E_0^2 = E_{01}^2+ E_{02}^2 + 2 E_{01} E_{02} \cos\delta
	\end{align*}
wobei $\delta = \varphi_1-\varphi_2 - kz_1 + kz_2$ die Gesamtphasendifferenz zwischen den beiden Wellen ist. Diese wird im Fall des Mach-Zehnder-Interferometers durch einer Strahlteilung folgende unterschiedliche optische Weglängen $z_1$, $z_2$ erreicht. Wird in einen der beiden Strahlen ein Medium der Dicke $\Delta$ mit Brechungsindex $n_M$ eingebracht, wird die Phase dieses Strahls um 
	\begin{align*}
	k\Delta(n_M - n_0) = 2\pi\frac{\Delta}{\lambda}(n_M-n_0)
	\end{align*}
vergrößert. Aus dem entstehenden Interferenzmuster kann durch Vergleich mit einem anderen (z.B. ohne das Medium) die relative Dickenänderung bestimmt werden.