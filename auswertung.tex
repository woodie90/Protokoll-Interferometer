\section{Auswertung}

Die Auswertung erfolgte in MathCad anhand der per Kamera aufgenommenen Bilder.

Zunächst wurde sich für eine qualitativ geeignete Bildzeile mit möglichst wenig Störungen entschieden. Es folgte ein Auslesen des Interferenzmusters über die Helligkeitswerte, um ein von der Bildspalte abhängiges Signal $P(s)$ zu erhalten; hierfür wurden geeignete Filter zur Elimination der Störungen verwendet. Aus dem erhaltenen (reellen) Signal wurde per Hilberttransformation das analytische Signal $P_c(s) = P(s) + \mathrm j \mathcal H\left\lbrace P(s) \right\rbrace$ generiert, das nun Phaseninformationen $\varphi(s)=\arg P_c(s)$ enthielt. Anschließend wurde aus der Phasendifferenz $\Delta\varphi(s)$  zwischen den beiden Signalen die Dickendifferenz $\Delta\varphi\lambda_0 / 2\pi n_M$ errechnet. Die jeweils letzten Diagramme der Anhänge zeigen den Verlauf der Dickendifferenz über die Breite des Laserstrahls.

In diesem Versuch wurden drei Objekte vermessen:

\paragraph*{Handelsübliche Brille (\texttt{brille2.bmp})}

\paragraph*{Laserschutzbrille (\texttt{laserbrille3.bmp})}

\paragraph*{Linse (\texttt{linse4.bmp})}